\documentclass{sigchi}

% Use this section to set the ACM copyright statement (e.g. for
% preprints).  Consult the conference website for the camera-ready
% copyright statement.

% Copyright
\CopyrightYear{2017}
%\setcopyright{acmcopyright}
\setcopyright{acmlicensed}
%\setcopyright{rightsretained}
%\setcopyright{usgov}
%\setcopyright{usgovmixed}
%\setcopyright{cagov}
%\setcopyright{cagovmixed}
% DOI
\doi{http://dx.doi.org/10.475/123_4}
% ISBN
\isbn{123-4567-24-567/08/06}
%Conference
\conferenceinfo{CHI'16,}{May 07--12, 2016, San Jose, CA, USA}
%Price
\acmPrice{\$15.00}

% Use this command to override the default ACM copyright statement
% (e.g. for preprints).  Consult the conference website for the
% camera-ready copyright statement.

%% HOW TO OVERRIDE THE DEFAULT COPYRIGHT STRIP --
%% Please note you need to make sure the copy for your specific
%% license is used here!
% \toappear{
% Permission to make digital or hard copies of all or part of this work
% for personal or classroom use is granted without fee provided that
% copies are not made or distributed for profit or commercial advantage
% and that copies bear this notice and the full citation on the first
% page. Copyrights for components of this work owned by others than ACM
% must be honored. Abstracting with credit is permitted. To copy
% otherwise, or republish, to post on servers or to redistribute to
% lists, requires prior specific permission and/or a fee. Request
% permissions from \href{mailto:Permissions@acm.org}{Permissions@acm.org}. \\
% \emph{CHI '16},  May 07--12, 2016, San Jose, CA, USA \\
% ACM xxx-x-xxxx-xxxx-x/xx/xx\ldots \$15.00 \\
% DOI: \url{http://dx.doi.org/xx.xxxx/xxxxxxx.xxxxxxx}
% }

% Arabic page numbers for submission.  Remove this line to eliminate
% page numbers for the camera ready copy
% \pagenumbering{arabic}

% Load basic packages
\usepackage{balance}       % to better equalize the last page
\usepackage{graphics}      % for EPS, load graphicx instead 
\usepackage[T1]{fontenc}   % for umlauts and other diaeresis
\usepackage{txfonts}
\usepackage{mathptmx}
\usepackage[pdflang={en-US},pdftex]{hyperref}
\usepackage{color}
\usepackage{booktabs}
\usepackage{textcomp}

% Some optional stuff you might like/need.
\usepackage{microtype}        % Improved Tracking and Kerning
% \usepackage[all]{hypcap}    % Fixes bug in hyperref caption linking
\usepackage{ccicons}          % Cite your images correctly!
% \usepackage[utf8]{inputenc} % for a UTF8 editor only

% If you want to use todo notes, marginpars etc. during creation of
% your draft document, you have to enable the "chi_draft" option for
% the document class. To do this, change the very first line to:
% "\documentclass[chi_draft]{sigchi}". You can then place todo notes
% by using the "\todo{...}"  command. Make sure to disable the draft
% option again before submitting your final document.
\usepackage{todonotes}

% Paper metadata (use plain text, for PDF inclusion and later
% re-using, if desired).  Use \emtpyauthor when submitting for review
% so you remain anonymous.
\def\plaintitle{Gaze and Touch}
\def\plainauthor{First Author, Second Author, Third Author,
  Fourth Author, Fifth Author, Sixth Author}
\def\emptyauthor{}
\def\plainkeywords{Authors' choice; of terms; separated; by
  semicolons; include commas, within terms only; required.}
\def\plaingeneralterms{Documentation, Standardization}

% llt: Define a global style for URLs, rather that the default one
\makeatletter
\def\url@leostyle{%
  \@ifundefined{selectfont}{
    \def\UrlFont{\sf}
  }{
    \def\UrlFont{\small\bf\ttfamily}
  }}
\makeatother
\urlstyle{leo}

% To make various LaTeX processors do the right thing with page size.
\def\pprw{8.5in}
\def\pprh{11in}
\special{papersize=\pprw,\pprh}
\setlength{\paperwidth}{\pprw}
\setlength{\paperheight}{\pprh}
\setlength{\pdfpagewidth}{\pprw}
\setlength{\pdfpageheight}{\pprh}

% Make sure hyperref comes last of your loaded packages, to give it a
% fighting chance of not being over-written, since its job is to
% redefine many LaTeX commands.
\definecolor{linkColor}{RGB}{6,125,233}
\hypersetup{%
  pdftitle={\plaintitle},
% Use \plainauthor for final version.
%  pdfauthor={\plainauthor},
  pdfauthor={\emptyauthor},
  pdfkeywords={\plainkeywords},
  pdfdisplaydoctitle=true, % For Accessibility
  bookmarksnumbered,
  pdfstartview={FitH},
  colorlinks,
  citecolor=black,
  filecolor=black,
  linkcolor=black,
  urlcolor=linkColor,
  breaklinks=true,
  hypertexnames=false
}

% create a shortcut to typeset table headings
% \newcommand\tabhead[1]{\small\textbf{#1}}

% End of preamble. Here it comes the document.
\begin{document}

\title{\plaintitle}

\numberofauthors{2}
\author{%
  \alignauthor{Armin Mokhtarian\\
    \affaddr{Aachen, Germany}\\
    \email{armin.mokhtarian@rwth-aachen.de}}\\
  \alignauthor{Oliver Nowak\\
    \affaddr{for Submission}\\
    \affaddr{City, Country}\\
    \email{e-mail address}}\\
}

\maketitle

\begin{abstract}

\end{abstract}

\category{H.5.m.}{Information Interfaces and Presentation
  (e.g. HCI)}{Miscellaneous} \category{See
  \url{http://acm.org/about/class/1998/} for the full list of ACM
  classifiers. This section is required.}{}{}

\keywords{\plainkeywords}

\section{Introduction}
Process from mouse input to touch and gaze
\subsection{Gaze}
\begin{itemize}
  \item Reasons for gaze input
  \begin{itemize}
    \item Primary sensory organ
    \item Very fast
    \item No fatigue
    \item "Why shouldn't I interact directly with a target, when I'm already look at it?"
  \end{itemize}
  \item Selection techniques (also presenting example approaches)
  \begin{itemize}
    \item Dwell Time
    \item Blinking
  \end{itemize}
  \item Problems with gaze
  \begin{itemize}
    \item Technology
    \item Not precise (eye jittery)
    \item No muscle memory
    \item Midas Touch problem
    \item Time based Selection (Too early gaze change which unintentionally aborts selection)
    \item Distraction (Eye is by nature a input organ, so it is also reacting on events in the peripheral vision)
    \item Intention (Unintended starring at the screen, should not be recognized by the system as target selection)
  \end{itemize}  
  \item "What about Fitts' Law?"
\end{itemize}

\subsection{Touch}
\begin{itemize}
  \item Pro
  \begin{itemize}
    \item Direct input
    \item Widespread \& familiar input method
    \item Well-known gestures
  \end{itemize}
  \item Contra
  \begin{itemize}
    \item Fat Finger problem for small targets or on small screens
    \item Reachability (How to interact on distant screens? (attention sharing) and How to interact with a target which is not directly in front of me (e.g. on tabletops)?)
    \item Speed (Is it faster than gaze?)
  \end{itemize}
 \end{itemize}

\section{Combining gaze with other modalities }
(Here we will present the different approaches/principles and cover the following topics) 

\begin{itemize}
  \item Transition from contras of mouse/touch/gaze only to gaze and touch
  \item Presenting different techniques
  \begin{itemize}
    \item Different principles: 'Gaze selects, touch manipulates' , Cursor-Warping, Gaze-Shifting etc.
    \item Applications for Gaze+Touch ( Interaction on distant displays/wall-sized displays, RST, Just speed up, Fitt’s Law)
  \end{itemize}
 \end{itemize}

\section{Evaluation}
\begin{itemize}
  \item More precise comparison of different techniques (not only Gaze+Touch methods, but also comparison to touch-only, gaze-only,...)
  \begin{itemize}
    \item Speed
    \item Comfort
    \item Accuracy
    \item Mental demanding?
    \item User satisfaction 
    \item ... 
  \end{itemize}
  \item Contra
  \begin{itemize}
    \item Fat Finger problem for small targets or on small screens
    \item Reachability (How to interact on distant screens? (attention sharing) and How to interact with a target which is not directly in front of me (e.g. on tabletops)?)
    \item Speed (Is it faster than gaze?)
  \end{itemize}
 \end{itemize}
\section{Conclusion}



% BALANCE COLUMNS
%\balance{}

% REFERENCES FORMAT
% References must be the same font size as other body text.
\bibliographystyle{SIGCHI-Reference-Format}
\bibliography{sample}
\end{document}











%----------------------------------------
%Kleines Bild
\begin{figure}
\centering
  \includegraphics[width=0.9\columnwidth]{figures/sigchi-logo}
  \caption{Insert a caption below each figure. Do not alter the
    Caption style.  One-line captions should be centered; multi-line
    should be justified. }~\label{fig:figure1}
\end{figure}


~\cite{acm_categories,ethics,Klemmer:2002:WSC:503376.503378}.

%halbwegs schicke tabelle 
\begin{table}
  \centering
  \begin{tabular}{l r r r}
    % \toprule
    & & \multicolumn{2}{c}{\small{\textbf{Test Conditions}}} \\
    \cmidrule(r){3-4}
    {\small\textit{Name}}
    & {\small \textit{First}}
      & {\small \textit{Second}}
    & {\small \textit{Final}} \\
    \midrule
    Marsden & 223.0 & 44 & 432,321 \\
    Nass & 22.2 & 16 & 234,333 \\
    Borriello & 22.9 & 11 & 93,123 \\
    Karat & 34.9 & 2200 & 103,322 \\
    % \bottomrule
  \end{tabular}
  \caption{Table captions should be placed below the table. We
    recommend table lines be 1 point, 25\% black. Minimize use of
    table grid lines.}~\label{tab:table1}
\end{table}
%%% Local Variables:
%%% mode: latex
%%% TeX-master: t
%%% End:
